\documentclass{article}
\begin{document}
\title{Synced Play \\ User Manual}
\author{Kevin Johnson}
\date{Last updated: \today}
\maketitle{}

\newpage

\tableofcontents

\newpage

\section{Introducing Synced Play}

Synced Play is a program developed to let users playback sound effects
at certain times during a theatrical performance. It is designed to
playback sounds with very low overhead (so as to make playback much snappier)
and to offer the option to loop sounds until told to stop. It also has a
cues feature that allows the user to pre-program the show during rehearsals,
and then simply step through the cues during the actual performances.

\newpage

\section{Sounds}

Each sound that you use in your show must be imported and associated with
a keyword.

\subsection{Keywords}

The keyword is a single word that you will use to refer to the sound for
the rest of the show.

Keywords must be a single word (no spaces) and contain only alphabetical
letters. Keywords can not be changed after you choose them, so choose
them wisely!

\subsection{Importing Sounds}

To import a sound, open the ``Sounds'' menu and select ``Import Audio''. Then
you must select an audio file\footnote{Currently only WAV and AU files
are supported} and then decide upon a keyword for the sound.

\newpage

\section{Actions}

Actions are things that can happen.\footnote{Who would have guessed!} In
Synced Play, actions are sentances that usually take the form ``verb keyword''

There are five different actions in Synced Play right now:

\subsection{Play Action}

The play action is arguably the most fundamental action in the program.
It allows you to play a sound once.

It takes the form ``play keyword'', where keyword is the keyword of the 
sound that you'd like to play. For example ``play doorbell'' would play
the sound with the keyword ``doorbell'' once.

\subsection{Loop Action}

The loop action is very closely related to the play action. It's almost
the same, but instead of playing a sound once, it will play the sound until
it is stopped.\footnote{Right now you don't know how to stop it. If you
start to loop a sound right now, it would loop forever! MWAHAHA!}

It takes the form ``loop keyword'', where keyword is the keyword of the 
sound that you'd like to loop. For example ``loop ringtone'' would play
the sound with the keyword ``ringtone'' until stopped\footnote{by an
unbalanced force...\footnotemark}.
\footnotetext{Oops, this isn't my physics homework!}

\subsection{Stop Action}

The stop action simply stops the playback of a sound effect.\footnote{It
is the savior of any of you who got trapped by the loop action} \footnote{
Note that if you stop sound effect A, it will not stop sound effect B,
but if you have two sound effect A's playing, it will stop both of them.}

It takes the form ``stop keyword'', where keyword is the keyword of the 
sound that you'd like to stop. For example ``stop ringtone'' will
stop any currently playing sound effects with the keyword, ``ringtone''.

\subsection{Panic Action}

The panic action is a little bit different. It is used to stop playback
of all sounds immediately. It's named, because it's what you want to use
if all of a sudden your computer starts to make a ton of sound.\footnote{
That is, sound coming out of Synced Play. If the sound's coming out of some
other program, or the computer hardware itself, you're out of luck.} Also,
it doesn't require a keyword.

It takes the form ``panic''. Yep, that's it. Just ``panic''.

\subsection{Goto Action}

The goto action is slightly different too, because it deals with cues
instead of with sounds. The goto action will set the ``on deck'' cue to
the specified cue. \footnote{If this doesn't make sense right now, it will
when you read the Cues section}

It takes the form ``goto cue-number'', where cue-number is the number of the
cue that you want to be on deck.

\newpage

\section{Cues}

Cues are basically a way to have actions happen in a certain order. A cue
has a ``Description'' and some ``Actions''. 

\subsection{Creating Cues}

To create a cue, open the ``Cues'' menu and select ``New Cue''. Give the cue
a description (so you know what it does later) and enter some actions. You
can enter one or more actions, each on a different line.

For example, you may have a part in your show where a baby had been softly
babbling up to that point, but then begins to scream. For the description
you might enter ``Baby begins to scream'', and your actions might be:
\begin{verbatim}
stop babyBabble
start babyScream
\end{verbatim}
In this way, the babbling would stop at the same time as the screaming
starts allowing a seamless transition.

\subsection{Using Cues}

Once you have your show cue'd out, you probably want to actually play it
back.\footnote{At some point at least, maybe you want to run through it
a couple of times to make sure you got it right, or not.} To playback
your cues, direct your attention to the bottom half of Synced Play. You
may have noticed that it lists all your cues. One of your cues has a star
next to it. This is the cue that is ``on deck''. If you press the ``TAB'' key
the ``on deck'' will be run, and the cue that is after it will then be
``on deck''. If you need to go forwards or backwards in the cue list without
actually running any cues, you can use the ``$<$'' key to go back, and the
``$>$'' key to go forwards.

\newpage

\section{Graphical Interface}

You've used programs before. I'm not going to tell you about menus and
such, but I will tell you whats different about Synced Play.

The program has three different parts.\footnote{Excluding menus but I'm not
going to explain those}

\subsection{The Sound Table}

The top part is the sound table. It lists all of
the loaded sounds and what their keywords are. If you right click on it
there will be a context menu with some things you can do.\footnote{I'm
not going to tell you what things you can do. I'll leave that as an exercise
to the reader.}

\subsection{The Cue Table}

The middle part is the cue table. It lists all of the cues that are
currently loaded. If you right click on it there will be a context menu
with some things you can do.\footnote{Again, another exercise for the reader}

\subsection{The Command Line}

The bottom part is the command line. It's a single line text box, so you
might not have even noticed it yet. You can directly type an action in
there and when you press ``ENTER'' it will be run.

\newpage

\section*{Thank You}

That's it. Thanks for reading this long document. It took me a while to 
write it.\footnote{But the footnotes helped keep me sane} Hopefully you
kinda know how to use Synced Play now.\footnote{If not, you're on your
own!}

\end{document}
